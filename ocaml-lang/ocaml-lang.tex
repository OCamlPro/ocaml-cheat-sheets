\documentclass[10pt,landscape]{article}
\usepackage{multicol}
\usepackage{calc}
\usepackage{ifthen}
\usepackage[landscape]{geometry}
\usepackage{epsfig}

\ifthenelse{\lengthtest { \paperwidth = 11in}}
	{ \geometry{top=.5in,left=.5in,right=.5in,bottom=.5in} }
	{\ifthenelse{ \lengthtest{ \paperwidth = 297mm}}
		{\geometry{top=1cm,left=1cm,right=1cm,bottom=1cm} }
		{\geometry{top=1cm,left=1cm,right=1cm,bottom=1cm} }
	}
\pagestyle{empty}
\makeatletter
\renewcommand{\section}{\@startsection{section}{1}{0mm}%
                                {-1ex plus -.5ex minus -.2ex}%
                                {0.5ex plus .2ex}%x
                                {\normalfont\large\bfseries}}
\renewcommand{\subsection}{\@startsection{subsection}{2}{0mm}%
                                {-1explus -.5ex minus -.2ex}%
                                {0.5ex plus .2ex}%
                                {\normalfont\normalsize\bfseries}}
\renewcommand{\subsubsection}{\@startsection{subsubsection}{3}{0mm}%
                                {-1ex plus -.5ex minus -.2ex}%
                                {1ex plus .2ex}%
                                {\normalfont\small\bfseries}}
\makeatother

% Define BibTeX command
\def\BibTeX{{\rm B\kern-.05em{\sc i\kern-.025em b}\kern-.08em
    T\kern-.1667em\lower.7ex\hbox{E}\kern-.125emX}}

% Don't print section numbers
\setcounter{secnumdepth}{0}


\setlength{\parindent}{0pt}
\setlength{\parskip}{0pt plus 0.5ex}


% -----------------------------------------------------------------------

\begin{document}

\setlength{\premulticols}{1pt}
\setlength{\postmulticols}{1pt}
\setlength{\multicolsep}{1pt}
\setlength{\columnsep}{2em}

\hspace{-2em}
\begin{tabular}{l}
\vspace{0cm}\epsfig{file=../licence/by-sa,width=24mm}\\
\end{tabular}\vspace{-2mm}
\hfill
\hfill
\Large{\textbf{The OCaml Language}}
\hfill
\hfill
\scriptsize
OCaml v. 3.12.0 ---
\today{} ---
OCamlPro SAS
({\bf http://www.ocamlpro.com/})
\\
\hrule~\\
\raggedright
\footnotesize
\begin{multicols}{3}

\subsection{Syntax}

Implementations are in \verb!.ml! files, interfaces are in \verb!.mli!
files.\\
 Comments can be nested, between delimiters \verb!(*!...\verb!*)!\\
Integers: \verb!123!, \verb!1_000!, \verb!0x4533!, \verb!0o773!, \verb!0b1010101!\\
Chars: \verb!'a'!, \verb!'\255'!, \verb!'\xFF'!, \verb!'\n'!
\hfill
Floats: \verb!0.1!, \verb!-1.234e-34!\\
\subsection{Data Types}

\begin{tabular}{ll}
\verb!unit! & Void, takes only one value: \verb!()!\\
\verb!int! & Integer of either 31 or 63 bits, like \verb!32! \\
\verb!int32! & 32 bits Integer, like \verb!32l! \\
\verb!int64! & 64 bits Integer, like \verb!32L! \\
\verb!float! & Double precision float, like \verb!1.0! \\
\verb!bool! & Boolean, takes two values: \verb!true! or \verb!false!\\
\verb!char! & Simple ASCII characters, like 'A'\\
\verb!string! & Strings of chars, like \verb!"Hello"!\\
\verb!'a list! & Lists, like \verb!head :: tail! or \verb![1;2;3]!   \\
\verb!'a array! & Arrays, like \verb![|1;2;3|]! \\
$t_1$ \verb!*! ... \verb!*! $t_n$& Tuples, like \verb!(1,"foo", 'b')!
\end{tabular}

\subsection{Constructed Types}

\begin{tabular}{ll}
\verb!type record =! & new record type \\
~~~ \verb!{!~~~~~~~~~~~~ \emph{field1} \verb!: bool;! & immutable field \\
~~~ \verb!  mutable! \emph{field2} \verb!: int; }! & mutable field \\
\end{tabular}
\begin{tabular}{ll}
\verb!type enum =! & new variant type \\
~~~ \verb!| Constant! & Constant constructor \\
~~~ \verb!| Param of string! & Constructor with arg \\
~~~ \verb!| Pair of string * int! & Constructor with args \\
\end{tabular}

\subsection{Constructed Values}

\begin{tabular}{ll}
\verb!let r = { field1 = true; field2 = 3; }! \\
\verb!let r' = { r with field1 = false }!\\
\verb!r.field2 <- r.field2 + 1;!\\
\verb!let c = Constant!\\
\verb!let c' = Param "foo"!\\
\verb!let c'' = Pair ("bar",3)!\\
\end{tabular}

\subsection{References, Strings and Arrays}

\begin{tabular}{ll}
\verb!let x = ref 3! & integer reference (mutable) \\
\verb!x := 4! & reference assignation \\
\verb&print_int !x;& & reference access \\
\verb!s.[0]! & string char access \\
\verb!s.[0] <- 'a'! & string char modification \\
\verb!t.(0)! & array element access \\
\verb!t.(0) <- x! & array element modification \\
\end{tabular}

\subsection{Imports --- Namespaces}

\begin{tabular}{ll}
\verb!open Unix;;! & global open \\
\verb!let open Unix in! \emph{expr} & local open \\
\verb!Unix.(!\emph{expr}\verb!)! & local open \\
\end{tabular}

\vbox{
\subsection{Functions}

\begin{tabular}{ll}
\verb!let f x =! \emph{expr} & function with one arg \\
\verb!let rec f x =! \emph{expr} & recursive function \\
\hfill apply:& \verb!f x! \\
\verb!let f x y =! \emph{expr} & with two args \\
\hfill apply:& \verb!f x y! \\
\verb!let f (x,y) =! \emph{expr} & with a pair as arg\\
\hfill apply: & \verb!f (x,y)! \\
\verb!List.iter (fun x ->! \emph{e}\verb!) l! & anonymous function\\
\verb!let f= function None ->! \emph{act}& function definition\\
\verb!            | Some x ->! \emph{act}& \hfill by cases \\
\hfill apply: & \verb!f (Some x)! \\
\verb!let f ~str ~len =! \emph{expr} & with labeled args \\
\hfill apply: & \verb!f ~str:s ~len:10! \\
\hfill apply (for \verb!~str:str!): & \verb!f ~str ~len! \\
\verb!let f ?len ~str =! \emph{expr} & with optional arg (\verb!option!) \\
\verb!let f ?(len=0) ~str =! \emph{expr} & optional arg default \\
\hfill apply (with omitted arg): & \verb!f ~str:s ! \\
\hfill apply (with commuting): & \verb!f ~str:s ~len:12! \\
\hfill apply (\verb!len: int option!): & \verb!f ?len ~str:s! \\
\hfill apply (explicitely ommited): & \verb!f ?len:None ~str:s! \\
\verb!let f (x : int) =! \emph{expr} & arg has constrainted type \\
\verb!let f : 'a 'b. 'a*'b -> 'a!& function with constrainted\\
\verb!      = fun (x,y) -> x! & \hfill polymorphic type\\
\end{tabular}


\subsection{Modules}

\begin{tabular}{ll}
\verb!module M = struct! .. \verb!end! & module definition\\
\verb!module M: sig! .. \verb!end= struct! .. \verb!end! & module and signature\\
\verb!module M = Unix! & module renaming \\
\verb!include M! & include items from \\
\verb!module type Sg = sig! .. \verb!end! & signature definition\\
\verb!module type Sg = module type of M! & signature of module\\
\verb!let module M = struct! .. \verb!end in! ..  & local module \\
\verb!let m = (module M : Sg)! & to $1^{st}$-class module\\
\verb!module M = (val m : Sg)! & from $1^{st}$-class module\\
\verb!module Make(S: Sg) = struct! .. \verb!end! & functor \\
\verb!module M = Make(M')! & functor application \\
\end{tabular}

Module type items:\\
\verb!val!, \verb!external!, \verb!type!, \verb!exception!, \verb!module!, \verb!open!, \verb!include!, \verb!class!

\subsection{Pattern-matching}

\begin{tabular}{ll}
\verb!match! \emph{expr} \verb!with! \\
\verb!  |! \emph{pattern} \verb!->! \emph{action}\\
\verb!  |! \emph{pattern} \verb!when! \emph{guard} \verb!->! \emph{action}
& conditional case \\
\verb!  | _ ->! \emph{action} & default case\\
\verb!  | exception! \emph{exn} \verb!->! \emph{action} & try\&match ($\geq$4.02.0) \\
\end{tabular}
Patterns:\\
\begin{tabular}{ll}
\verb!| Pair (x,y) ->! & variant pattern \\
\verb!| { field = 3; _ } ->! & record pattern \\
\verb!| head :: tail ->! & list pattern \\
\verb!| [1;2;x] ->! & list-pattern \\
\verb!| (Some x) as y ->! & with extra binding \\
\verb!| (1,x) | (x,0) ->! & or-pattern \\
\end{tabular}
}

\vbox{
\subsection{Conditionals}

\begin{tabular}{|c|c|l|}\hline
Structural  & Physical & \\
\verb!=!  & \verb!==! & Polymorphic Equality \\
\verb!<>! & \verb&!=& & Polymorphic Inequality \\
\end{tabular}

Polymorphic Generic Comparison Function: \verb!compare!
\begin{tabular}{|l|c|c|c|}\hline
                          & x $<$ y & x = y & x $>$ y \\
\verb!compare x y! &  -1    &   0   &   1 \\
\end{tabular}

Other Polymorphic Comparisons : \verb!>!, \verb!>=!, \verb!<!, \verb!<=!
}

\subsection{Loops}

\begin{verbatim}
while cond do ... done;
for var = min_value to max_value do ... done;
for var = max_value downto min_value do ... done;
\end{verbatim}

\subsection{Exceptions}

\begin{tabular}{p{3.5cm}cp{3.5cm}}
\verb!exception MyExn! && new exception \\
\verb!exception MyExn of t * t'!&& same with arguments\\
\verb!exception MyFail = Failure! && rename exception with args\\
\verb!raise MyExn! && raise an exception\\
\verb!raise (MyExn (!\emph{args}\verb!))! && raise with args\\
\verb!try!~\emph{expression} \verb!with Myn -> ...! & & catch \verb!MyException! if raised in \emph{expression} \\
\end{tabular}

\subsection{Objects and Classes}

\begin{tabular}{ll}
\verb!class virtual foo x = !& virtual class with arg \\
\verb! let y = x+2 in! & init before object creation\\
\verb! object (self: 'a)! & object with self reference\\
\verb!  val mutable variable = x! & mutable instance variable \\
\verb!  method get = variable! & accessor \\
\verb!  method set z =!\\
\verb!     variable <- z+y! & mutator\\
\verb!  method virtual copy : 'a! & virtual method\\
\verb!  initializer! & init after object creation\\
\verb!   self#set (self#get+1)!& \\
\verb! end! &  \\
\verb!class bar = !&  non-virtual class\\
\verb! let var = 42 in! & class variable\\
\verb! fun z -> object! & constructor argument \\
\verb& inherit foo z as super& & inheritance and ancestor reference\\
\verb& method&{\bf \verb&!&}\verb! set y = ! & method explicitely overriden\\
\verb!    super#set (y+4)! & access to ancestor \\
\verb! method copy = {< x = 5 >}! & copy with change \\
\verb!end! & \\
\verb!let obj = new bar 3! & new object \\
\verb!obj#set 4; obj#get!  & method invocation \\
\verb!let obj = object! .. \verb!end! & immediate object \
\end{tabular}

\subsection{Polymorphic variants}


\begin{tabular}{ll}
\verb!type t = [ `A | `B of int ]! & closed variant \\
\verb!type u = [ `A | `C of float ]! & \\
\verb!type v = [ t | u | ]! & union of variants \\
\verb!let f : [< t ] -> int = function! & argument must be\\
\verb!  | `A -> 0 | `B n -> n! & \hfill a subtype of \verb!t!\\
\verb!let f : [> t ] -> int = function! & \verb!t! is a subtype \\
\verb!  | `A -> 0 | `B n -> n | _ -> 1! & \hfill of the argument \\
\end{tabular}

\end{multicols}


\end{document}

\subsection{Misc}

