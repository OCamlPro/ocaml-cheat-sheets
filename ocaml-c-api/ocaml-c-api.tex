\documentclass[10pt,landscape]{article}
\usepackage{multicol}
\usepackage{calc}
\usepackage{ifthen}
\usepackage[landscape]{geometry}
\usepackage{epsfig}

\ifthenelse{\lengthtest { \paperwidth = 11in}}
	{ \geometry{top=.5in,left=.5in,right=.5in,bottom=.5in} }
	{\ifthenelse{ \lengthtest{ \paperwidth = 297mm}}
		{\geometry{top=1cm,left=1cm,right=1cm,bottom=1cm} }
		{\geometry{top=1cm,left=1cm,right=1cm,bottom=1cm} }
	}
\pagestyle{empty}
\makeatletter
\renewcommand{\section}{\@startsection{section}{1}{0mm}%
        {-1ex plus -.5ex minus -.2ex}%
        {0.5ex plus .2ex}%x
        {\normalfont\large\bfseries}}
\renewcommand{\subsection}{\@startsection{subsection}{2}{0mm}%
        {-1explus -.5ex minus -.2ex}%
        {0.5ex plus .2ex}%
        {\normalfont\normalsize\bfseries}}
\renewcommand{\subsubsection}{\@startsection{subsubsection}{3}{0mm}%
        {-1ex plus -.5ex minus -.2ex}%
        {1ex plus .2ex}%
        {\normalfont\small\bfseries}}
\makeatother

% Define BibTeX command
\def\BibTeX{{\rm B\kern-.05em{\sc i\kern-.025em b}\kern-.08em
 T\kern-.1667em\lower.7ex\hbox{E}\kern-.125emX}}

% Don't print section numbers
\setcounter{secnumdepth}{0}


\setlength{\parindent}{0pt}
\setlength{\parskip}{0pt plus 0.5ex}


% -----------------------------------------------------------------------

\begin{document}

\setlength{\premulticols}{1pt}
\setlength{\postmulticols}{1pt}
\setlength{\multicolsep}{1pt}
\setlength{\columnsep}{3mm}

\hspace{-2em}
\begin{tabular}{l}
\vspace{0cm}\epsfig{file=../licence/by-sa,width=24mm}\\
\end{tabular}\vspace{-2mm}
\hfill
\hfill
\Large{\textbf{OCaml --- C API}}
\hfill
\hfill
\scriptsize
OCaml v. 3.12.0 ---
\today{} ---
Copyright \copyright\ 2011 OCamlPro SAS
---
{\bf http://www.ocamlpro.com/}\\
\hrule~\\
\raggedright
\footnotesize
\begin{multicols}{3}

\vbox{
\section{Declaring Primitives}

1 to 5 arguments:\\
{\tt external} \emph{ocamlfun} {\tt :} \emph{ocamltype} {\tt =} \verb!"!c\_stub\verb!"!

more than 5 arguments:\\
{\tt external} \emph{ocamlfun} {\tt :} \emph{ocamltype} {\tt =} \verb!"!tupled\_C\_stub\verb!"!  \verb!"!c\_stub\verb!"!

The C\_stub function has as many arguments as the OCaml function, while
tupled\_C\_stub has the C prototype:\\

{\tt CAMLprim value} tuple\_c\_stub\verb!(value * argv, int argn)!

where \verb!argn! is the number of values stored in \verb!argv!.

\section{Include Files}
\begin{tabular}{ll}
{\tt caml/mlvalues.h} & value type, and conversion macros \\
{\tt caml/alloc.h} & allocation functions \\
{\tt caml/memory.h} & memory-related functions and macros \\
{\tt caml/fail.h} & functions for raising exceptions \\
{\tt caml/callback.h} & callback from C to OCaml \\
{\tt caml/custom.h} & operations on custom blocks \\
{\tt caml/intext.h} & de/serialization for custom blocks \\
\end{tabular}

\section{Linking with C code}

\subsection{Static Linking}

\verb!-custom!

\subsection{Dynamic Linking}

\verb!.so! and \verb!.dll!

\section{Value Representation}

The {\tt value} type is either (1) an unboxed integer, (2) a pointer
to a block inside the OCaml heap, or (3) a pointer outside the OCaml
heap (not scanned by GC).

A {\tt value} pointing within OCaml heap should always point to the
beginning of an OCaml block.

Integers are encoded as $(n$\verb!<<!$ 1) + 1$, and so limited to 31
bits on 32 bits arch, 63 bits on 64 bits arch.

Blocks in the heap are arrays of {\tt value}s, preceded by a header
{\tt value}. The header {\tt value} contains the number of {\tt
  value}s in the array, a tag on one character, and 3 bits for
garbage-collection.

Interesting tag values:\\
\begin{tabular}{ll}
0..\verb!No_scan_tag! -- 1 & Standard OCaml Blocks. \\
\verb!Closure_tag! & A closure, i.e. code pointers plus env. \\
\verb!String_tag! & An OCaml string. \\
\verb!Double_tag! & An OCaml float. \\
\verb!Double_array_tag! & An Array of OCaml floats. \\
\verb!Abstract_tag! & A block that should not be scanned. \\
\verb!Custom_tag! & A block with custom functions. \\
\end{tabular}

}

\subsection{Representations}


\section{The 6 GC Safety Rules}

\paragraph{Rule 1:}
A function that has parameters or local variables of type {\tt value}
must begin with a call to one of the {\tt CAMLparam} macros and return
with {\tt CAMLreturn}, {\tt CAMLreturn0}, or {\tt CAMLreturnT}.

\paragraph{Rule 2:}
 Local variables of type value must be declared with one of the {\tt
   CAMLlocal} macros. Arrays of values are declared with {\tt
   CAMLlocalN}. These macros must be used at the beginning of the
 function, not in a nested block.

\paragraph{Rule 3:}
Assignments to the fields of structured blocks must be done with the
\verb!Store_field! macro (for normal blocks) or
\verb!Store_double_field! macro (for arrays and records of
floating-point numbers). Other assignments must not use
\verb!Store_field!  nor \verb!Store_double_field!.

\paragraph{Rule 4:}
Global variables containing values must be registered with the garbage
collector using the \verb!caml_register_global_root! function.

\paragraph{Rule 5:}
 After a structured block (a block with tag less than
 \verb!No_scan_tag!) is allocated with the low-level functions, all
 fields of this block must be filled with well-formed values before
 the next allocation operation. If the block has been allocated with
 \verb!caml_alloc_small!, filling is performed by direct assignment to
 the fields of the block:

\verb!Field(v, n) = vn ;!

If the block has been allocated with \verb!caml_alloc_shr!, filling is
performed through the \verb!caml_initialize! function:

\verb!caml_initialize(&Field(v, n), vn );!

\paragraph{Rule 6:}
Direct assignment to a field of a block, as in

\verb!Field(v, n) = w;!

is safe only if \verb!v! is a block newly allocated by
\verb!caml_alloc_small!; that is, if no allocation took place between
the allocation of \verb!v! and the assignment to the field. In all
other cases, never assign directly.  If the block has just been
allocated by \verb!caml_alloc_shr!, use
\verb!caml_initialize! to
assign a value to a field for the first time:

\verb!caml_initialize(&Field(v, n), w);!

Otherwise, you are updating a field that previously contained a
well-formed value; then, call the \verb!caml_modify function!:

\verb!caml_modify(&Field(v, n), w);!

\section{The ``noalloc'' flag}

\section{Macros}

\verb!Is_long(v)! is true if value v is an immediate integer, false otherwise\\
\verb!Is_block(v)! is true if value v is a pointer to a block, and false if it is an immediate integer.\\

\verb!Val_long(l)! C long int to OCaml int.\\
\verb!Long_val(v)! OCaml int to C long int.\\
\verb!Val_int(i)! C int to OCaml int.\\
\verb!Int_val(v)! OCaml int to C int.\\
\verb!Val_bool(x)! C truth value to OCaml bool.\\
\verb!Bool_val(v)! OCaml bool to C truth value.\\
\verb!Val_true!, \verb!Val_false! OCaml true and false.\\

\verb!Wosize_val(v)! returns the size of the block v, in words, excluding the header.\\
\verb!Tag_val(v)! returns the tag of the block v.\\
\verb!Field(v, n)! returns the value contained in the nth field of the structured block v. Fields are numbered from 0 to \verb!Wosize_val(v) -- 1!.\\
\verb!Store_field(b, n, v)! stores the value v in the field number n of value b, which must be a structured block.\\
\verb!caml_string_length(v)! returns the length (number of characters) of the string v. \\
\verb!String_val(v)! returns a pointer to the first byte of the string
v, with type char *. This pointer is a valid C string: there is a null
character after the last character in the string.  However, Caml
strings can contain embedded null characters, that will confuse the
usual C functions over strings.\\
\verb!Double_val(v)! returns the floating-point number contained in
value v, with type double.\\
\verb!Double_field(v, n)! returns the nth element of the array of
floating-point numbers v (a block tagged \verb!Double_array_tag!).\\
\verb!Store_double_field(v, n, d)! stores the double precision
floating-point number d in the nth element of the array of
floating-point numbers v. \\

\subsection{Allocation Functions}

\verb!caml_alloc(n, t)! returns a fresh block of size n with tag t. If t is less than \verb!No_scan_tag!, then
the fields of the block are initialized with a valid value in order to satisfy the GC constraints.

\verb!caml_alloc_tuple(n)! returns a fresh block of size n words, with tag 0.

\verb!caml_alloc_string(n)! returns a string value of length n
characters. The string initially contains garbage.

\verb!caml_copy_string(s)! returns a string value containing a copy of
the null-terminated C string s (a char *).

\verb!caml_copy_double(d)! returns a floating-point value initialized
with the double d.


\section{Functions}

\section{Callbacks from C to OCaml}

\end{multicols}
\end{document}
