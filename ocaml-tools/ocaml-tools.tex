\documentclass[10pt,landscape]{article}
\usepackage{multicol}
\usepackage{calc}
\usepackage{ifthen}
\usepackage[landscape]{geometry}
\usepackage{epsfig}

\ifthenelse{\lengthtest { \paperwidth = 11in}}
	{ \geometry{top=.5in,left=.5in,right=.5in,bottom=.5in} }
	{\ifthenelse{ \lengthtest{ \paperwidth = 297mm}}
		{\geometry{top=1cm,left=1cm,right=1cm,bottom=1cm} }
		{\geometry{top=1cm,left=1cm,right=1cm,bottom=1cm} }
	}
\pagestyle{empty}
\makeatletter
\renewcommand{\section}{\@startsection{section}{1}{0mm}%
                                {-1ex plus -.5ex minus -.2ex}%
                                {0.5ex plus .2ex}%x
                                {\normalfont\large\bfseries}}
\renewcommand{\subsection}{\@startsection{subsection}{2}{0mm}%
                                {-1explus -.5ex minus -.2ex}%
                                {0.5ex plus .2ex}%
                                {\normalfont\normalsize\bfseries}}
\renewcommand{\subsubsection}{\@startsection{subsubsection}{3}{0mm}%
                                {-1ex plus -.5ex minus -.2ex}%
                                {1ex plus .2ex}%
                                {\normalfont\small\bfseries}}
\makeatother

% Define BibTeX command
\def\BibTeX{{\rm B\kern-.05em{\sc i\kern-.025em b}\kern-.08em
    T\kern-.1667em\lower.7ex\hbox{E}\kern-.125emX}}

% Don't print section numbers
\setcounter{secnumdepth}{0}


\setlength{\parindent}{0pt}
\setlength{\parskip}{0pt plus 0.5ex}


% -----------------------------------------------------------------------

\begin{document}
\hspace{-2em}
\begin{tabular}{l}
\vspace{0cm}\epsfig{file=../licence/by-sa,width=24mm}\\
\end{tabular}\vspace{-2mm}
\hfill
\hfill
\Large{\textbf{OCaml Standard Tools}}
\hfill
\hfill
\scriptsize
OCaml v. 3.12.0 ---
\today{} ---
Copyright \copyright\ 2011 OCamlPro SAS
---
{\bf http://www.ocamlpro.com/}\\
\hrule~\\
\raggedright
\footnotesize
\begin{multicols}{3}

% multicol parameters
% These lengths are set only within the two main columns
%\setlength{\columnseprule}{0.25pt}
\setlength{\premulticols}{1pt}
\setlength{\postmulticols}{1pt}
\setlength{\multicolsep}{1pt}
\setlength{\columnsep}{2pt}

\section{Standard Tools}

\verb!.opt! tools are the same tools, compiled in native-code, thus
much faster.

\noindent\begin{tabular}{p{20mm}p{50mm}}
\verb!ocamlopt![\verb!.opt!]  & native-code compiler \\
\verb!ocamlc![\verb!.opt!]     & bytecode compiler \\
\verb!ocaml!                      & interactive bytecode toplevel  \\
\verb!ocamllex![\verb!.opt!]  & lexer compiler \\
\verb!ocamlyacc!                & parser compiler \\
\verb!ocamldep![\verb!.opt!]   & dependency analyser \\
\verb!ocamldoc!                  & documentation generator \\
\verb!ocamlrun!                  & bytecode interpreter \\
\end{tabular}

\section{Compiling}

A unit interface must be compiled before its implementation.  Here,
\verb!ocamlopt! can replace \verb!ocamlc! anywhere to target asm.

~\hspace{-1em}\noindent
\begin{tabular}{ll}
\verb!ocamlc -c test.mli! & compile an interface \\
\verb!ocamlc -c test.ml! & compile an implementation \\
\verb!ocamlc -a -o lib.cma test.cmo! & generate a library \\
\verb!ocamlc -o prog test.cmo! & generate an executable \\
\multicolumn{2}{l}{
{\tt ocamlopt -shared -o p.cmxs test.cmx}
\hfill generate a plugin}\\
\end{tabular}

\subsection{Generic Arguments}

\noindent\begin{tabular}{p{25mm}p{50mm}}
\verb!-config! & print config and exit\\
\verb!-c! & do not link \\
\verb!-o!~\emph{target} & specify the target to generate\\
\verb!-a! & build a library  \\
\verb!-pp!~\emph{prepro} & use a preprocessor (often \verb!camlp4!) \\
\verb!-I!~\emph{directory}     & search directory for dependencies \\
\verb!-g!     & add debugging info \\
\verb!-annot!     & generate source navigation information \\
\verb!-i!     & print inferred interface \\
\verb!-thread! & generate thread-aware code\\
\verb!-linkall!     & link even unused units \\
\verb!-nostdlib! & do not use installation directory \\
\verb!-nopervasives! & do not autoload \verb!Pervasives! \\
\end{tabular}

\subsection{Linking with C}

\noindent\begin{tabular}{p{25mm}p{50mm}}
\verb!-cc!~\emph{gcc} & use as C compiler/linker\\
\verb!-cclib!~\emph{option} & pass option to the C linker\\
\verb!-ccopt!~\emph{option} & pass option to C compiler/linker \\
\verb!-output-obj! & link, but output a C object file \\
\verb!-noautolink! & do not automatically link C libraries\\
\end{tabular}

\subsection{Errors and Warnings}

Warnings default is \verb!+a-4-6-7-9-27..29!
\noindent\begin{tabular}{p{25mm}p{50mm}}
\verb!-w!~\emph{wlist} & set or unset warnings\\
\verb!-warn-errors!~\emph{wlist} & set or unset warnings as errors\\
\verb!-warn-help! & print description of warnings\\
\verb!-rectypes! & allow arbitrarily recursive types \\
\end{tabular}

\subsection{Native-code Specific Arguments}

\noindent\begin{tabular}{p{20mm}p{50mm}}
\verb!-p!                & compile or link for profiling with \verb!gprof! \\
\verb!-inline!~\emph{size} & set maximal function size for inlining \\
\verb!-unsafe!   & remove array bound checks \\
\end{tabular}

\subsection{Bytecode Specific Arguments}

\noindent\begin{tabular}{p{30mm}p{50mm}}
\verb!-custom!          & link with runtime and C libraries \\
\verb!-make-runtime! & generate a pre-customized runtime \\
\verb!-use-runtime!~\emph{runtime}  & use \emph{runtime} instead of \verb!ocamlrun! \\
\end{tabular}

\subsection{Packing Arguments}

\noindent\begin{tabular}{p{30mm}p{45mm}}
\verb!-pack -o!~\emph{file.cmo/.cmx}  & pack several units in one unit \\
\verb!-c -for-pack!~\emph{File} & compile unit to be packed into \emph{File} \\
\end{tabular}

\section{Interactive Toplevel}
Use \verb!;;! to terminate and execute what you typed.\\
Building your own: \verb!ocamlmktop -o unixtop unix.cma!\\
\begin{tabular}{ll}
\verb!#load "lib.cma";;! & load a compiled library/unit \\
\verb!#use "file.ml";;! & compile and run a source file\\
\verb!#directory "dir";;! & add directory to search path\\
\verb!#trace!~\emph{function}\verb!;;! & trace calls to function \\
\verb!#untrace!~\emph{function}\verb!;;! & stop tracing calls to function \\
\verb!#quit;;! & quit the toplevel \\
\end{tabular}

\section{System Variables}

\begin{tabular}{ll}
\verb!OCAMLLIB! & Installation directory \\
\verb!OCAMLRUNPARAM! & Runtime settings (e.g. \verb!b,s=256k,v=0x015!)\\
\end{tabular}
\begin{tabular}{l|llll}
Flags & \verb!p! & ocamlyacc parser trace & \verb!b! & print backtrace \\
 & \verb!i! & major heap increment & \verb!s! & minor heap size \\
 & \verb!O! & compaction overhead  & \verb!o! & space overhead \\
 & \verb!s! & stack size  & \verb!h! & initial heap size \\
 & \verb!v! & GC verbosity \\
\end{tabular}

\section{Files Extensions}
\begin{tabular}{p{5mm}p{21mm}|p{15mm}p{21mm}}
\multicolumn{2}{c}{Sources} &\multicolumn{2}{c}{Objects}\\\hline
.ml & implementation & .cmo & bytecode object \\
& & .cmx + .o & asm object \\
.mli & interface & .cmi & interface object \\
.mly & parser & .cma & bytecode library \\
.mll & lexer & .cmxa + .a & native library \\
&& .cmxs & native plugin \\
\end{tabular}

%\section{Interactive Toplevel}
%
%\section{Bytecode Interpreter}

\section{Generating Documentation}

\begin{tabular}{l}
Generate documentation for source files:\\
\verb!ocamldoc!~\emph{format}~\verb!-d!~\emph{directory}~\emph{sources.mli}\\
where \emph{format} is:
\noindent\begin{tabular}{|p{15mm}p{55mm}}
\verb!-html!   & 	Generate HTML \\
\verb!-latex! & 	Generate LaTeX \\
\verb!-texi! & 	Generate TeXinfo \\
\verb!-man! & 		Generate man pages \\
\end{tabular}
\end{tabular}

\vbox{
~\\
{\large\bf Parsing} ~~~~~\verb!ocamlyacc grammar.mly!\\
will generate \verb!grammar.mli! and \verb!grammar.ml! from the grammar specification.

\noindent\begin{tabular}{p{5mm}p{65mm}}
\verb!-v! & generates \verb!grammar.output! file with debugging info \\
\end{tabular}

\begin{tabular}{p{17mm}p{50mm}}
\vspace{-2cm}
\begin{verbatim}
%{
  header
%}
  declarations
%%
  rules
%%
  trailer
\end{verbatim}
&
\begin{tabular}{p{45mm}}\hline
\noindent Declarations:\\
\begin{tabular}{ll}
\verb!%token!~\emph{token} & \verb!%left!~\emph{symbol}\\
\verb!%token <!\emph{type}\verb!>!~\emph{token} & \verb!%right!~\emph{symbol}\\
\verb!%start!~\emph{symbol} & \verb!%nonassoc!~\emph{symbol}\\
\verb!%type <!\emph{type}\verb!>!~\emph{symbol}\\
\end{tabular}\\\hline
Rules:\\
\begin{tabular}{l}
nonterminal \verb!:!\\
    \emph{symbol} ... \emph{symbol} \verb!{! \emph{action} \verb!}!\\
  \verb!|! ...\\
  \verb!|! \emph{symbol} ... \emph{symbol} \verb!{! \emph{action} \verb!} ;!\\
\end{tabular}\\ \hline
\end{tabular}
\end{tabular}

{\large\bf Lexing} ~~~~~\verb!ocamllex lexer.mll!\\
will generate \verb!lexer.ml! from the lexer specification.

\noindent\begin{tabular}{p{5mm}p{65mm}}
\verb!-v! & generates \verb!lexer.output! file with debugging info \\
\end{tabular}

\hspace{-1em}\begin{tabular}{p{45mm}p{23mm}}
\begin{tabular}{l}
\{ header \}\\
\verb!let! ident \verb!=! regexp ...\\
\verb!rule! entrypoint \emph{args} \verb!=!\\
~~~\verb!parse! regexp \verb!{! \emph{action} \}\\
~~~~~~\verb!|! ...\\
~~~~~~\verb!|! regexp \verb!{! \emph{action} \verb!}!\\
\verb!and! entrypoint \emph{args} \verb!=!\\
~~~\verb!parse! ...\\
\verb!and! ...\\
\verb!{! trailer \}\\
\end{tabular} &
\verb!Lexing.lexeme lexbuf! in \emph{action} to get the current
token.
\end{tabular}


\section{Computing Dependencies}

\verb!ocamldep! can be used to automatically compute dependencies.  It
takes in arguments all the source files (.ml and .mli), and some
standard compiler arguments:
\noindent\begin{tabular}{p{15mm}p{55mm}}
\verb!-pp! \emph{prepro} & call a preprocessor \\
\verb!-I! \emph{dir}      & search directory for dependencies \\
\verb!-modules!        & print modules instead of Makefile format \\
\verb!-slash!           & use $\backslash{}$ instead of / \\
\end{tabular}

\section{Generic Makefile Rules}

\begin{verbatim}
.SUFFIXES: .mli .mll .mly .ml .cmo .cmi .cmx
.ml.cmo :
        ocamlc -c $(OFLAGS) $(INCLUDES) $<
.mli.cmi :
        ocamlc -c $(OFLAGS) $(INCLUDES) $<
.ml.cmi :
        ocamlc -c $(OFLAGS) $(INCLUDES) $<
.ml.cmx :
        ocamlopt -c $(OFLAGS) $(INCLUDES) $<
.mll.ml :
        ocamllex $(OLEXFLAGS) $<
.mly.ml :
        ocamlyacc $(OYACCFLAGS) $<
.mly.mli:
        ocamlyacc $(OYACCFLAGS) $<
\end{verbatim}
}

\end{multicols}
\end{document}


\section{Optional Warnings/Errors}
For \verb!-w! and \verb!-warn-error!:

\begin{tabular}{lp{7cm}}
1  & Suspicious-looking start-of-comment mark\\
2  & Suspicious-looking end-of-comment mark\\
3  & Deprecated syntax\\
4  & Matching remains complete even if type is extended\\
5  & Partially applied function\\
6  & Label omitted in function application\\
7  & Methods overridden in defining class\\
8  & Partial match: missing cases in pattern-matching\\
9  & Missing fields in a record pattern\\
10  & Non-unit first expr in sequence (\verb!-strict-sequence!)\\
11  & Redundant case in a pattern matching.\\
12  & Redundant sub-pattern in a pattern-matching.\\
13  & Override of an instance variable.\\
14  & Illegal backslash escape in a string constant.\\
15  & Private method made public implicitly.\\
16  & Unerasable optional argument.\\
17  & Undeclared virtual method.\\
18  & Non-principal type.\\
19  & Type without principality.\\
20  & Unused function argument.\\
21  & Non-returning statement.\\
22  & Camlp4 warning\\
23  & Useless record \verb!with! clause\\
24  & Invalid module name\\
25  & Pattern-matching exhaustiveness cannot be checked\\
26  & Unused variable not starting with \verb!_! in \verb!let! or \verb!as!\\
27  & Unused variable not starting with \verb!_! in arguments\\
28  & Constant variant with a wildcard pattern as argument\\
29  & Unescaped end-of-line in a string constant\\
30  & Same constructors defined in mutually recursive types\\
\end{tabular}

SRCS= x.mly y.mll z.ml
TMP_SRCS= x.ml x.mli y.ml
INCLUDES=-I library
all: byte
byte: prog.byte
opt: prog.opt
prog.byte: $(CMO_OBJS)
        ocamlc -o prog.byte library/lib.cma $(CMO_OBJS)
prog.opt: $(CMX_OBJS)
        ocamlopt -o prog.opt library/lib.cmxa $(CMX_OBJS)
clean:
        rm -f *.cm? *~ *.o *.a *.cmx?
depend: $(TMP_SRCS)
        ocamldep $(INCLUDES) $(TMP_SRCS) $(SRCS) > .depend
-include .depend
